\section{Attenuation}
Associated external model: \texttt{attenuate.py}

Attenuation evaluation for quantity of interest $u$ with input parameters $Y=[y_1,\ldots,y_N]$:
\begin{equation}
u(Y) = \prod_{n=1}^N e^{-y_n/N}.
\end{equation}
This is the solution to the exit strength of a monodirectional, single-energy beam of neutral particles
incident on a unit length material divided into $N$ sections with independently-varying absorption cross
sections.  This test is useful for its analytic statistical moments as well as difficulty to represent
exactly using polynomial representations.

\subsection{Uniform}
Let all $y_n$ be uniformly distributed between 0 and 1.  The first two statistical moments are:
\subsubsection{mean}
\begin{align}
\expv{u(Y)} &=\int_{0}^1 dY \rho(Y)u(Y),\notag \\
  &=\int_{0}^1 dy_1\cdots\int_{0}^1 dy_N \prod_{n=1}^N e^{-y_n/N},\notag \\
  &=\left[ \int_{0}^1 dy e^{-y/N}\right]^N,\\
  &=\left[\left(-Ne^{-y/N}\right)\bigg|_0^1\right]^N,\notag \\
  &=\left[N\left(1-e^{-1/N}\right)\right]^N. \notag
\end{align}
\subsubsection{variance}
\begin{align}
\expv{u(Y)^2} &= \int_{0}^1 dY \rho(Y)u(Y), \notag \\
  &=\int_{0}^1 dy_1\cdots\int_{0}^1 dy_N \frac{1}{1^N} \left(\prod_{n=1}^N e^{-y_n/N}\right)^2,\notag \\
  &=\left[\left(\int_{0}^1 dy\ e^{-2y/N} \right)\right]^N,\notag \\
  &=\left[\left(\frac{N}{2}e^{-2y/N} \right)\bigg|_{0}^1 \right]^N, \\
  &=\left[\frac{N}{2}\left(1-e^{-2/N}\right)\right]^N.\notag \\
\text{var}[u(Y)] &= \expv{u(Y)^2}-\expv{u(Y)}^2, \notag \\
  &= \left[\frac{N}{2}\left(1-e^{-2/N}\right)\right]^N - \left[N\left(1-e^{-1/N}\right)\right]^{2N}.
\end{align}
\subsubsection{numeric values}
Some numeric values for the mean and variance are listed below for several input cardinalities $N$.
\begin{table}[h!]
\centering
\begin{tabular}{c|c|c}
$N$ & mean & variance \\ \hline
2 & 0.61927248698470190 & 0.01607798775751018 \\
4 & 0.61287838657652779 & 0.00787849640356994 \\
6 & 0.61075635579491642 & 0.00520852933409887
\end{tabular}
\end{table}

\subsection{Multivariate Normal}
Let $Y$ be $N$-dimensional, and have a multivariate normal distribution:
\begin{equation}
Y \thicksim N(\mu,\Sigma)
\end{equation}
with $N$-dimensional mean vector $\mu=[\mu_{y_1},\mu_{y_2},\ldots,\mu_{y_N}]$, and $N X N$ covariance matrix:
\begin{equation}
\Sigma = [Cov[y_i,y_j]],i = 1,2,\ldots,N; j = 1,2,\ldots,N
\end{equation}

To be simplicity, we assume there are no correlations between the input parameters. Then, the covariance matrix can be written
as:
\begin{equation}
\Sigma =
\begin{pmatrix}
\sigma_{y_1}^2 & 0 &\ldots & 0 \\
0 & \sigma_{y_2}^2 &\ldots & 0 \\
\vdots &\vdots &\ddots & \vdots \\
0 & 0 & \ldots & \sigma_{y_N}^2\\
\end{pmatrix}
\end{equation}
where $\sigma_{y_i}^2 = Cov[y_i,y_i]$, for $i = 1,2,\ldots,N$. Based on this assumption, the first two statistical moments are:
\subsubsection{mean}
\begin{align}
\expv{u(Y)} &=\int_{-\infty}^\infty dY \rho(Y)u(Y) \notag \\
  &=\int_{-\infty}^\infty dy_1 (1/\sqrt{2 \pi \sigma_{y_1}}e^{-\frac{(y_1-\mu_{y_1})^2}{2\sigma_{y_1}^2}})\cdots\int_{-\infty}^\infty dy_N  (1/\sqrt{2 \pi \sigma_{y_N}}e^{-\frac{(y_N-\mu_{y_N})^2}{2\sigma_{y_N}^2}})\prod_{n=1}^N e^{-y_n/N} \\
  &=\prod_{n=1}^N e^{\frac{\sigma_{y_i}^2}{2n^2}-\frac{\mu_{y_i}}{n}}. \notag
\end{align}
\subsubsection{variance}
\begin{align}
\text{var}[u(Y)]=\expv{(u(Y)-\expv{u(Y)})^2} &= \int_{-\infty}^\infty dY \rho(Y)(u(Y)-\expv{u(Y)})^2 \notag \\
  &=\int_{-\infty}^\infty dy_1 (1/\sqrt{2 \pi \sigma_{y_1}}e^{-\frac{(y_1-\mu_{y_1})^2}{2\sigma_{y_1}^2}})\\
  &\cdots\int_{-\infty}^\infty dy_N  (1/\sqrt{2 \pi \sigma_{y_N}}e^{-\frac{(y_N-\mu_{y_N})^2}{2\sigma_{y_N}^2}})(\prod_{n=1}^N e^{-y_n/N}-\expv{u(Y)})^2 \notag\\
  &=\prod_{n=1}^N e^{\frac{2 \sigma_{y_i}^2}{n^2}-\frac{2\mu_{y_i}}{n}}. \notag
\end{align}
\subsubsection{numeric values}
For example, for given mean $\mu = [0.5, -0.4, 0.3, -0.2, 0.1]$, and covariance
\begin{equation}
\Sigma =
\begin{pmatrix}
0.64 & 0 & 0 & 0 & 0 \\
0 & 0.49 & 0 & 0 & 0 \\
0 & 0 & 0.09 & 0 & 0 \\
0 & 0 & 0 & 0.16 & 0 \\
0 & 0 & 0 & 0 & 0.25 \\
\end{pmatrix}
\end{equation}
The mean and variance can computed using previous equation, and the results are:
\begin{equation}
\expv{u(Y)} = 0.97297197488624509
\end{equation}
\begin{equation}
\text{var}{u(Y)} = 0.063779804051749989
\end{equation}

\subsection{Changing lower, upper bounds}
A parametric study can be made by changing the lower and upper bounds of the material opacities.

The objective is to determine the effects on the exit strength $u$ of a beam impinging on a
unit-length material subdivided into two materials with opacities $y_1, y_2$.  The range of values for
these opacities varies from lower bound $y_\ell$ to higher bound $y_h$, and the bounds are always
the same for both opacities.

We consider evaluating the lower and upper bounds
on a grid, and determine the expected values for the opacity means and exit strength.

The analytic values for the exit strength expected value depends on the lower and upper bound
as follows:
\begin{align}
  \bar u(y_1,y_2) &= \int_{y_\ell}^{y_h}\int_{y_\ell}^{y_h} \left(\frac{1}{y_h-y_\ell}\right)^2
    e^{-(y_1+y_2)/2} dy_1 dy_2, \\
    &= \frac{4e^{-y_h-y_\ell}\left(e^{y_h/2}-e^{y_\ell/2}\right)^2}{(y_h-y_\ell)^2}.
\end{align}

Numerically, the following grid points result in the following expected values:

\begin{table}[h!]
\centering
\begin{tabular}{c c|c|c}
$y_\ell$ & $y_h$ & $\bar y_1=\bar y_2$ & $\bar u$ \\ \hline
0.00 & 0.50 & 0.250 & 0.782865 \\
0.00 & 0.75 & 0.375 & 0.695381 \\
0.00 & 1.00 & 0.500 & 0.619272 \\
0.25 & 0.50 & 0.375 & 0.688185 \\
0.25 & 0.75 & 0.500 & 0.609696 \\
0.25 & 1.00 & 0.625 & 0.541564 \\
0.50 & 0.50 & 0.500 & 0.606531 \\
0.50 & 0.75 & 0.625 & 0.535959 \\
0.50 & 1.00 & 0.750 & 0.474832
\end{tabular}
\end{table}

