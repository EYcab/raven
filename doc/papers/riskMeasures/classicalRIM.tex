\section{Classical RIMs}
\label{sec:classicalRIMs}

% In classical PRA methods, for any basic event, the most used RIMs measures are: 
% Risk Achievement Worth (RAW), Risk Reduction Worth (RRW), Birnbaum (B) and 
% Fussell-Vesely (FV)~\cite{flemingRiskImportance}. 
% All these RIMs are calculated by determining three values based on core damage 
% frequency (CDF):
% \begin{itemize}
%   \item $R_0$: nominal CDF
%   \item $R_i^-$: CDF for basic event i assuming perfectly reliable
%   \item $R_i^+$: CDF for basic event i assuming it has failed
% \end{itemize}
% 
% Once these three values are determined, then the RIMs are calculated~\cite{} as 
% follows for each basic event $i$:
% \begin{align} 
%   RAW_i &= \frac{R_i^+}{R_0}    \\
%   RRW_i &= \frac{R_0}{R_i^-}    \\
%   B_i &= R_i^+-R_i^-            \\
%   FV_i &= \frac{R_0-R_i^-}{R_0} 
% \end{align}
% 
% Note the four RIMs listed above is not exhaustive: in literature it is possible to 
% find additional RIMs such as the 
% Differential Importance Measure (DIM)~\cite{BorgonovoApostolakis}. 
% Since, the scope of this paper is tight to risk-informed application of 10CFR50.69, 
% we will focus this paper only on the four RIMs listed above.

Nuclear industry PRA codes such as SAPHIRE can calculate the following seven different basic event importance 
measures for each basic event for the respective fault tree, accident sequence, or end state:
\begin{itemize}
  \item Fussell-Vesely (FV)
  \item Risk Increase Ratio (RIR)
  \item Risk Increase Difference (RID)
  \item Risk Reduction Ratio (RRR)
  \item Risk Reduction Difference (RRD)
  \item Birnbaum (B)
  \item Uncertainty Importance
\end{itemize}   
  
The FV importance measure indicates the fraction of the minimal cut set upper bound 
(or sequence frequency, core damage frequency) contributed by the cut sets containing the basic event of interest. 
It is calculated in SAPHIRE Version 8 as $FV = F(i) / F(x)$
where:
\begin{itemize}
  \item $F(x)$ is the value of all the minimal cut sets evaluated with the basic event probabilities at their mean value
  \item $F(i)$ is the value of all the minimal cut sets that contain the basic event $i$.
\end{itemize}

    
The RIR or RID importance measure indicates the increase (in relative ratio changes or in actual differences) of the minimal cut set upper bound (or sequence frequency, core damage frequency) when the basic event of interest has failed (i.e., the basic event failure probability is 1.0). 

The RIR importance is often called Risk Achievement Worth (RAW) in industry. The risk increase importance measures are calculated in SAPHIRE Version 8 as follows: $RIR = F(1)/F(x)$ and $RID=F(1)-F(x)$
where:
\begin{itemize} 
  \item $F(x)$ is the value of all the minimal cut sets evaluated with the basic event probabilities at their mean value.
  \item $F(1)$ is the value of all the minimal cut sets evaluated with the probability of the basic event of interest set to 1.0.
\end{itemize}

The RRR or RRD importance measure indicates the reduction (in relative ratio changes or in actual differences) of the minimal cut set upper bound (or sequence frequency, core damage frequency) if the basic event of interest never fails (i.e., the basic event failure probability is 0.0). The Risk Reduction Ratio importance is also often called RRW in industry. The risk decrease importance measures are calculated in SAPHIRE Version 8 as follows:
$RRR =F(x)/F(0)$ and $RRD=F(x)-F(0)$ where:
\begin{itemize} 
  \item $F(x)$ is the value of all the minimal cut sets evaluated with the basic event probabilities at their mean values.
  \item $F(0)$ is the value of all the minimal cut sets evaluated with the probability of the basic event of interest set to 0.0.
\end{itemize}

The Birnbaum importance measure is an indication of the sensitivity of the minimal cut set upper bound (or sequence frequency, core damage frequency) with respect to the basic event of interest. It is calculated as $B=F(1)-F(0)$
\begin{itemize} 
  \item F(1) = value of all the minimal cut sets evaluated with the probability of the basic event of interest set to 1.0.
  \item F(0) = value of all the minimal cut sets evaluated with the probability of the basic event of interest set to 0.0.
 \end{itemize}
  
The Uncertainty Importance measure is an indication of the contribution of the  basic event of interest uncertainty to the total output uncertainty. This importance measure is not widely used and is not discussed in further detail.


