\section{Measuring similarities}
\label{sec:similarities}

An important modeling choice when dealing with time series is the type of similarity 
metric employed to measure similarity among time series: a distance metric. 
Similarly to any distance metric defined in an euclidean space, 
a distance metric $d(S,Q)$ between two time series $S$ and $Q$ must obey the following rules:
\begin{equation}
  \begin{cases}
    d(S,S) = 0 \\
    d(S,Q) = d(Q,S) \\ 
    d(S,Q) = 0  \Longleftrightarrow  S=Q \\ 
    d(S,Q) \leq d(S,K) + d(K,Q) 
  \end{cases}
  \label{eq:distanceRules}
\end{equation}

When dealing with time series, the two metrics are the most commonly used: Euclidean and 
Dynamic Time Warping (DTW) distance. These distances are described in the next two subsections for
the univariate case, i.e., two time series Q and S where their continuous part has $M=1$. 
The more generic case, i.e., multivariate case, can be easily expanded from what is shown below.

\subsection{Euclidean Distance}
\label{sec:euclidean}

Given two univariate time series S and Q having identical length (i.e., $T_S=T_Q$) the Euclidean 
distance $d_2(S,Q)$ is defined as:
\begin{equation}
  d_2(S,Q) = \sqrt{ \sum_{t=0}^{T_s} (x_1^S(t)-x_1^Q(t))}
  \label{eq:euclidean}
\end{equation}

\begin{figure}
    \centering
    \includegraphics[scale=0.3]{L2.pdf}
    \caption{}
    \label{fig:2Danalogy}
\end{figure} 

Note that this distance metric requires identical 

\subsection{DTW Distance}
\label{sec:dtw}


This distance can be viewed as a natural extension of the Euclidean distance applied to time series~\cite{}. 
Given two univariate time series $S$ and $Q$ having length $T_S$ and $T_Q$ respectively, the distance value 
$d_DTW (S,Q)$ is calculated by following these two steps:
\begin{enumerate}
  \item Create a matrix $D=[d(i,j)]$ having dimensionality $T_S \times T_Q$ where each element of $D$ (see Fig.~\ref{} for 
        the time series shown in Fig.~\ref{}) is calculated as $d(i,j)=(x_1^S[i]-x_1^Q[j])^2$ for
        $i=1,\ldots,T_S$ and $j=1,\ldots,T_Q$.
  \item Search a continuous path $w_k|_1^K$ in the matrix $D$ that, starting from $(i,j)=(0,0)$, it ends at 
        $(i,j)=(T_S,T_Q)$ and it minimizes the sum of all the $K$ elements $w_k=(d(i,j))_k$ of this
       path (see blue line in Fig.~\ref{}):
      \begin{equation}
        d_DTW(S,Q) = min⁡(\sum_{k=1}^{K} w_k)
        \label{eq:dtw}
      \end{equation}       
       Each element of the path corresponds to a specific black segment in Fig.~\ref{}. This metric can capture 
       similarities between time series that are shifted in time.
\end{enumerate}

\begin{figure}
  \centering
  \begin{subfigure}{.5\textwidth}
    \centering
    \includegraphics[scale=0.25]{DTW_path.pdf}
  \end{subfigure}%
  \begin{subfigure}{.5\textwidth}
    \centering
    \includegraphics[scale=0.25]{DTW.pdf}
  \end{subfigure}
  \caption{.}
  \label{fig:DTW}
\end{figure}

