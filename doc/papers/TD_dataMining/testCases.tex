\section{Test Cases}
\label{sec:testCases}

\subsection{Pump Controller}
\label{sec:pumpController}

The first test case considers a pump controller model for a hypothetical simplified PWR model 
(see Figure 15) which consists of the following components: Reactor core (RX), Motor operated 
pump, Pump digital controller, Heat exchanger (HX).
This system is responsible to remove the decay heat generated from the core (RX) in order to 
avoid damage of the core itself. The objective is to maintain the temperature in the reactor 
core between 1500 C and 1600 C. The top-events are thus the following:
\begin{itemize}
  \item Success: final temperature between 1500 C and 1600 C
  \item Failure-high: final temperature greater than 3000 C
  \item Failure-low: final temperature lower than 1600 C
\end{itemize}

While we assumed that both the HS and the pump are perfectly reliable components (i.e., no 
failure can be introduced), using~\cite{} as a references, the digital pump controller 
reliability model has been performed using a continuous time Markov Chain formulation.

In more detail, the controller has been modeled using 4 states (see. Fig.~\ref{}):
\begin{itemize}
  \item State 0 - Operating: controller operating as designed
  \item State 1 - Failed closed: controller failed by sending close signal to pump (i.e., pump not running)
  \item State 2 - Failed stuck: controller failed by sending oldest valid signal to pump
  \item State 3 - Failed random: controller failed by sending close signal to pump
\end{itemize}
For the scope of this report we have assumed a constant (in time) failure rate $\lambda$ 
for all three transitions 
shown in Fig.~\ref{}. The scope of this exercise is to identify the impact of both
timing and type of failure on the system dynamics.

\begin{figure}
  \begin{subfigure}{.5\linewidth}
    \centering
    \includegraphics[scale=0.7]{controller.pdf}
  \end{subfigure}%
  \begin{subfigure}{.5\linewidth}
    \centering
    \includegraphics[scale=0.7]{Markov.pdf}
  \end{subfigure}
  \caption{Pump controller test case: scheme of the system considered and continuous time Markov model for the pump controller.}
  \label{fig:pumpController}
\end{figure}

In order to perform such analysis the model has been coded as a RAVEN external model (i.e., Python based code) 
which determine the temporal profile of core temperature give the two stochastic parameters: Pump controller 
failure time and Pump controller failure mode
The dynamic of the system has been modeled using basic mass and energy conservation laws so no effective 
engineering conclusions can be gathered by this example. An example of scenario where no pump failure occurs 
is shown in Fig.~\ref{}. Note that the transient has been divided along the temporal axis in 5 regions where:
\begin{itemize}
  \item Pump is ON in regions 1, 3 and 5
  \item Pump is OFF in regions 2 and 4
\end{itemize}

\begin{figure}
    \centering
    \centerline{\includegraphics[scale=0.3]{scenarioController.pdf}} 
    \caption{Pump controller: example of scenario where no pump failure occurs.}
    \label{fig:scenarioController}
\end{figure}

By using RAVEN we sampled the two stochastic parameters (see the histograms of these variables in Fig.~\ref{}) 
using a Monte-Carlo algorithm and generated 1500 simulations as shown in Fig.~\ref{}.

\begin{figure}
    \centering
    \centerline{\includegraphics[scale=0.4]{controllerAllScen.pdf}} 
    \caption{Pump controller: plot of the 1500 histories generated by RAVEN.}
    \label{fig:controllerAllScen}
\end{figure}

By observing only static values such as max or final temperature (see Fig.~\ref{}) it is not really possible to 
extract valuable information from the data set. 
For the analysis of this dataset we have chosen to use hierarchical clustering using Euclidean distance as 
distance metrics. The dendrogram obtained is shown in Fig.~\ref{}. From this dendrogram it is
clearly possible to identify 3 clusters which lead us to choose a separation level equal to 10; 
the plot of the scenarios colored by the clustering label value is shown in Fig.~\ref{}.

\begin{figure}
    \centering
    \centerline{\includegraphics[scale=0.4]{controllerInitialHist.pdf}} 
    \caption{Histograms of max (left) and final (right) temperature of the simulations shown in Fig.~\ref{}}
    \label{fig:controllerInitialHist}
\end{figure}

\begin{figure}
    \centering
    \centerline{\includegraphics[scale=0.4]{controllerDend1.pdf}} 
    \caption{Dendogram obtained using hierarchical clustering (euclidean distance) for the dataset 
             shown in Fig.~\ref{}}
    \label{fig:controllerDend1}
\end{figure}

\begin{figure}
    \centering
    \centerline{\includegraphics[scale=0.4]{controllerClusteredScen.pdf}} 
    \caption{Plot of the 1500 histories generated by RAVEN (see Fig.~\ref{}) colored based on the labels assigned 
             by the hierarchical clustering (see Fig.~\ref{}).}
    \label{fig:controllerClusteredScen}
\end{figure}

\begin{figure}
  \centering
  \begin{minipage}{.33\textwidth}
  \centering
  \includegraphics[width=\linewidth]{1-Clustered_HS_1_line.png}
  \end{minipage}\hfill
  \begin{minipage}{.33\textwidth}
  \centering
  \includegraphics[width=\linewidth]{1-HistTime_1_histogram.png}
  \end{minipage}\hfill
  \begin{minipage}{.33\textwidth}
  \centering
  \includegraphics[width=\linewidth]{1-HistMode_1_histogram.png}
  \end{minipage}
  \caption{Cluster 1 (see Fig.~\ref{}): plot of the histories (left), histograms of failure mode (center) 
           and failure time (right).}
  \label{fig:pump_cluster1}
\end{figure}

\begin{figure}
  \centering
  \begin{minipage}{.33\textwidth}
  \centering
  \includegraphics[width=\linewidth]{1-Clustered_HS_2_line.png}
  \end{minipage}\hfill
  \begin{minipage}{.33\textwidth}
  \centering
  \includegraphics[width=\linewidth]{1-HistTime_2_histogram.png}
  \end{minipage}\hfill
  \begin{minipage}{.33\textwidth}
  \centering
  \includegraphics[width=\linewidth]{1-HistMode_2_histogram.png}
  \end{minipage}
  \caption{Cluster 2 (see Fig.~\ref{}): plot of the histories (left), histograms of failure mode (center) 
           and failure time (right).}
  \label{fig:pump_cluster2}
\end{figure}

\begin{figure}
  \centering
  \begin{minipage}{.33\textwidth}
  \centering
  \includegraphics[width=\linewidth]{1-Clustered_HS_3_line.png}
  \end{minipage}\hfill
  \begin{minipage}{.33\textwidth}
  \centering
  \includegraphics[width=\linewidth]{1-HistTime_3_histogram.png}
  \end{minipage}\hfill
  \begin{minipage}{.33\textwidth}
  \centering
  \includegraphics[width=\linewidth]{1-HistMode_3_histogram.png}
  \end{minipage}
  \caption{Cluster 3 (see Fig.~\ref{}): plot of the histories (left), histograms of failure mode 
           (center) and failure time (right).}
  \label{fig:pump_cluster3}
\end{figure}

\begin{figure}
    \centering
    \centerline{\includegraphics[scale=0.4]{1-Clustered_HS_sub_line.png}} 
    \caption{Plot of the histories belonging to Cluster 1 (see Fig.~\ref{}) colored based on the 
             labels assigned by the hierarchical clustering (see Figure 69).}
    \label{fig:controllerDend1}
\end{figure}

The analysis of the obtained clusters is summarized in Fig.~\ref{} (Cluster 1), Fig.~\ref{} (Cluster 2), 
Fig.~\ref{} (Cluster 3). In order to describe the obtained results we refer also to Fig.~\ref{}:
\begin{itemize}
  \item Cluster 1 (see Fig.~\ref{}) contains a large number of scenarios that lead to both ``failure-high'' and 
        ``success'' top-events and no other particular information can be deduced
  \item Cluster 2 (see Fig.~\ref{}) and 3 (see Fig.~\ref{}8) contain scenarios where pump controller in 
        State 2 (i.e., stuck) while the pump was actually ON in region 1 and 3 respectively (Fig.~\ref{}). They
        are all leading to the “failure-low” top-event Given the fact that cluster 1 contains too much variety 
        of scenarios, we performed hierarchical clustering on just the scenarios belonging to Cluster 1:
        i.e., sub-clustering. The obtained dendrogram is shown in Fig.~\ref{}. In this case, 5 clusters 
        were obtained; the plot of the scenarios colored by the clustering label value is shown in Fig.~\ref{}.
\item 
\end{itemize}

The analysis of the obtained clusters is summarized in Fig.~\ref{} (Cluster 1), Fig.~\ref{} (Cluster 2), 
Fig.~\ref{} (Cluster 3), Fig.~\ref{} (Cluster 4) and Fig.~\ref{} (Cluster 5):
\begin{itemize}
  \item Clusters 1, 2, 4 and 5 (see Fig.~\ref{}, Fig.~\ref{}, Fig.~\ref{} and Fig.~\ref{} respectively) 
        contain scenarios where pump controller failed in any of the three modes (closed, stuck and
        random) but did not lead to the failure-high top event 
  \item Cluster 3 (see Fig.~\ref{}) contains scenarios where pump controller failed in only two modes 
        (closed and random). Note that:
  \begin{itemize}
    \item Pump failed in the first time region
    \item Controller failure random did not lead failure-high top event; however, controller failure closed 
          lead to very high core temperatures (including 3000 C)
   \end{itemize}
   This cluster contains scenarios leading to failure-high and success top events. Note that if controller 
   failure occurs prior to about 115 min that the simulation leads to failure-high top event.
\end{itemize}

In summary, using the analytical model data set we were able to gather the following information:
\begin{itemize}
  \item Failure-high top event can be reached if controller failure to state 1 (i.e., failure closed) occurs 
        prior to 115 min 
  \item Failure-low top event can be reached if controller failure to state 2 (i.e., failure stuck) occurs 
        in the time regions 1 and 3
  \item Controller failure to state 3 (i.e., failure random) do not lead to any failure top event independently 
        of the controller failure time
  \item For these controller failure events the core temperatures are between 1500 C and 3000 C:
  \begin{itemize}
    \item Controller failure to state 1 (i.e., failure closed) after to 115 min
    \item Controller failure to state 2 (i.e., stuck) in time regions 2, 4 and 5
  \end{itemize}
\end{itemize}

\begin{figure}
  \centering
  \begin{minipage}{.33\textwidth}
  \centering
  \includegraphics[width=\linewidth]{1-Clustered_HS_sub_1_line.png}
  \end{minipage}\hfill
  \begin{minipage}{.33\textwidth}
  \centering
  \includegraphics[width=\linewidth]{1-HistTimeSub_1_histogram.png}
  \end{minipage}\hfill
  \begin{minipage}{.33\textwidth}
  \centering
  \includegraphics[width=\linewidth]{1-HistModeSub_1_histogram.png}
  \end{minipage}
  \caption{Cluster 1 (see Fig.~\ref{}): plot of the histories (left), histograms of failure mode 
           (center) and failure time (right).}
  \label{fig:pump_cluster_1_1}
\end{figure}

\begin{figure}
  \centering
  \begin{minipage}{.33\textwidth}
  \centering
  \includegraphics[width=\linewidth]{1-Clustered_HS_sub_2_line.png}
  \end{minipage}\hfill
  \begin{minipage}{.33\textwidth}
  \centering
  \includegraphics[width=\linewidth]{1-HistTimeSub_2_histogram.png}
  \end{minipage}\hfill
  \begin{minipage}{.33\textwidth}
  \centering
  \includegraphics[width=\linewidth]{1-HistModeSub_2_histogram.png}
  \end{minipage}
  \caption{Cluster 2 (see Fig.~\ref{}): plot of the histories (left), histograms of failure mode 
           (center) and failure time (right).}
  \label{fig:pump_cluster_1_2}
\end{figure}

\begin{figure}
  \centering
  \begin{minipage}{.33\textwidth}
  \centering
  \includegraphics[width=\linewidth]{1-Clustered_HS_sub_3_line.png}
  \end{minipage}\hfill
  \begin{minipage}{.33\textwidth}
  \centering
  \includegraphics[width=\linewidth]{1-HistTimeSub_3_histogram.png}
  \end{minipage}\hfill
  \begin{minipage}{.33\textwidth}
  \centering
  \includegraphics[width=\linewidth]{1-HistModeSub_3_histogram.png}
  \end{minipage}
  \caption{Cluster 3 (see Fig.~\ref{}): plot of the histories (left), histograms of failure mode 
           (center) and failure time (right).}
  \label{fig:pump_cluster_1_3}
\end{figure}

\begin{figure}
  \centering
  \begin{minipage}{.33\textwidth}
  \centering
  \includegraphics[width=\linewidth]{1-Clustered_HS_sub_4_line.png}
  \end{minipage}\hfill
  \begin{minipage}{.33\textwidth}
  \centering
  \includegraphics[width=\linewidth]{1-HistTimeSub_4_histogram.png}
  \end{minipage}\hfill
  \begin{minipage}{.33\textwidth}
  \centering
  \includegraphics[width=\linewidth]{1-HistModeSub_4_histogram.png}
  \end{minipage}
  \caption{Cluster 4 (see Fig.~\ref{}): plot of the histories (left), histograms of failure mode 
           (center) and failure time (right).}
  \label{fig:pump_cluster_1_4}
\end{figure}


\begin{figure}
  \centering
  \begin{minipage}{.33\textwidth}
  \centering
  \includegraphics[width=\linewidth]{1-Clustered_HS_sub_5_line.png}
  \end{minipage}\hfill
  \begin{minipage}{.33\textwidth}
  \centering
  \includegraphics[width=\linewidth]{1-HistTimeSub_5_histogram.png}
  \end{minipage}\hfill
  \begin{minipage}{.33\textwidth}
  \centering
  \includegraphics[width=\linewidth]{1-HistModeSub_5_histogram.png}
  \end{minipage}
  \caption{Cluster 5 (see Fig.~\ref{}): plot of the histories (left), histograms of failure mode 
           (center) and failure time (right).}
  \label{fig:pump_cluster_1_5}
\end{figure}
