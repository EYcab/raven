\section{Test Case}
\label{sec:testCase}

The test case considered in this paper is a 3-loop PWR system of Westinghouse 
design with a large-dry containment.
The initiating event is a LLOCA: a double-ended guillotine break of the hot leg.

Under these accident conditions, the system experiences a sudden sub-cooled blowdown and 
primary system pressure drops from about 2200 psi down to the saturation pressure 
(about 1000 psi).

In order to compensate the large loss of coolant inventory into the vessel and prevent
core damage, two emergency core-cooling systems are employed: Accumulators and 
Low-Pressure Injection System (LPIS).

Accumulators are passive components consisting of pressurized water tanks that are 
employed at the beginning of the transient and can dump large inventory of sub-cooled water
into the vessel.

The LPIS is activated when primary system pressure falls below 980 psi and by using pumps
large amount of water are transfered from the Reactor Water Storage Tank (RWST) directly 
into the vessel.

The large amount of water that leaves the primary system and is collected in the containment
and, thus, its temperature and pressure increases. In order to cool it down, Containment
Sprays (CSs) are employed. Similarly to the LPIS, through pumps RWST water is sprayed from the 
top level of the containment. 

Once RWST is empty, both CSs and LPISs switch from injection mode to recirculation mode: 
water collected at the base of the containment and through the sump and is injected back 
into the vessel (through the LPI) and into the containment (through the CSs).


\section{Comparison Workflow}

The comparison between Classical (using SAPHIRE) and Dynamic PRA analysis (using RAVEN/RELAP5-3D)
has been performed following these steps:
\begin{enumerate}
	\item Simplify the SAPHIRE FT structure by grouping its basic events into macro basic events.
	\item Perform calculation of the ET-FT model: determine probability of each ET branch and the 
	      risk importance of each macro basic events
	\item Consider the SAPHIRE ET for the LLOCA initiating event and model the RELAP5-3D accident 
	      progression following consistently with the ET logic. 
	\item Construct the PWR logic based on the same macro basic events determined in Step 1: these
	      macro basic events constitute the stochastic variables sampled by RAVEN 
	\item Introduce time-dependent variables into the the PWR logic: this set of variables is also 
	      part of the stochastic variables sampled by RAVEN 
	\item Perform a dynamic analysis using RAVEN/RELAP5-3D for the system constructed in Steps 4 
	      and 5, and  determine probability of each ET branch and the risk importance of each
	      macro basic events
	\item Compare the results obtained in Steps 3 and 6
\end{enumerate}

Step 1 has been deemed necessary in order to limit the number of unnecessary RELAP5-3D simulation runs.
The rationale is that each basic event in the set of FTs are also stochastic variables sampled by RAVEN.
If all basic events are considered (about ????) then the sampling strategy would require a very large
number of RELAP5-3D simulation runs. The great majority of these runs would be identical since the
impact of these basic events on accident progression is identical.

The main target of this comparison is to identify commonalities and inconsistencies between classical 
and Dynamic PRA methods on the accident progression level (i.e., at the ET level) while the FT level is 
modeled in the same way for both methods.
