\section{Dynamic PRA Analysis}
\label{sec:RISMCanalysis}

The Dynamic PRA analysis have been performed using RAVEN and RELAP5-3D and employing 
both desktop and High Performing Computing (HPC) systems. The analyses performed for 
Case 1 and 2 were different in structure and they are described in detail in 
Sections~\label{sec:RISMC_1_analysis} and~\label{sec:RISMC_2_analysis} respectively.


\section{Case 1 Analysis}
\label{sec:RISMC_1_analysis}

Case 1 has been performed by following these steps:
\begin{enumerate}
	\item Perform the set of RAVEN/RELAP5-3D calculations. The variables of interests
	      are the following 
	      \begin{itemize}
	      	\item ACC\_1: Accumulator 1 failure
	      	\item ACC\_2: Accumulator 2 failure
	      	\item LPI\_A: LPI train A failure
	      	\item LPI\_B: LPI train B failure
	      	\item LPR\_A: LPR train A failure
	      	\item LPR\_B: LPR train A failure
	      \end{itemize}
	      Since these variables are discrete in nature (i.e., 0 or 1) we have explored all
	      possible combinations (using a Grid sampling strategy), i.e., $2^6=64$, 
	      by simulating using RELAP5-3D the system response for all 64 cases.
	\item Use the data generated in Step 1 to train a ROM. The ROM predict system outcome
	      (OK or CD) and maximum clad temperature given a combination of the six input 
	      variables listed above
	\item Perform a RAVEN analysis using the ROM determined in Step 2. The set of input 
	      variables are the ones listed in Table~\ref{}: the 18 macro basic events defined also 
	      in the SAPHIRE LLOCA ET/FT model. 
	      In order to create a connection between the 18 sampled variables and the 6 input 
	      variables of the ROM, a set of six RAVEN functions have been introduced in the analysis
	      so that, for each combination of the 18 sampled variables, the corresponding combination
	      of the six ROM input variables is determined.
	      Since the sampled variables are discrete in nature (True or False, 0 or 1), all possible
	      combinations (i.e., $2^18=262,144$) of the sampled variables have been simulated 
	      (using a Grid sampling strategy).
	      Note that the outcome of each combination is determined by evaluating the ROM instead 
	      of RELAP5-3D and, thus, computational time is strongly reduced. 
	\item Process the data generated in Step 3 and determine CD probability along with the risk
	      measure for each sampled variable.
\end{enumerate}

\section{Case 2 Analysis}
\label{sec:RISMC_2_analysis}

Case 2 has been performed by directly sampling RELAP5-3D using RAVEN without employing ROM. 
In addition to the 18 stochastic variables listed in Table~\ref{} an additional set of time-dependent
variables were added as indicated in Table~\ref{tab:case2NewVars}.
In this case, a Monte-Carlo sampling strategy have been used and a total of XXX RELAP5-3D runs were 
generated



