\section{PWR Stochastic Modeling}
\label{PWRstochasticModeling}

In the Dynamic PRA analysis, the stochastic modeling of the PWR system have been performed
for two different cases:
\begin{itemize}
	\item Case 1: set of RAVEN stochastic variables coincide with the set of macro basic events
	      defined in Section~\ref{sec:SAPHIREmodeling}
	\item Case 2: set of RAVEN stochastic variables include not only the set of macro basic events
	      defined in Section~\ref{sec:SAPHIREmodeling} but also includes time related variables. As an example, 
	      the failure to run of a component is modeled quantitatively by sampling not only if such 
	      event occurs but the failure time of such component is actually sampled.
\end{itemize}

In Case 1, the PWR models in both Dynamic and classical PRA methods are identical: there is a 
one-to-one connection between the stochastic elements (basic events vs. sampled variables) and
accident progression (system simulator vs. ET) of both methods.

In case 1 the set of macro basic events coincide with the set of stochastic parameters sampled
by RAVEN. While in SAPHIRE each macro basic event ($BE_i$) is characterized by probability value 
($p_i$), in RAVEN its corresponding stochastic variable ($v_i$) is modeled through a Bernoulli 
distribution $v_i \sim \operatorname{Bern}(p_i)$.
From this analysis we expect very similar results since the capabilities of Dynamic PRA methods
are not exploited.
However, this is a first test toward benchmarking Dynamic vs. classical PRA methods in order to 
understand modeling differences and impact on final results.

