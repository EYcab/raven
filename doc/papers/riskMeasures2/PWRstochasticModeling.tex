\section{PWR Stochastic Modeling}
\label{PWRstochasticModeling}

In the Dynamic PRA analysis, the stochastic modeling of the PWR system have been performed
for two different cases:
\begin{itemize}
	\item Case 1: set of RAVEN stochastic variables coincide with the set of macro basic events
	      defined in Section~\ref{}
	\item Case 2: set of RAVEN stochastic variables include not only the set of macro basic events
	      defined in Section~\ref{} but also includes time related variables. As an example, 
	      the failure to run of a component is modeled quantitatively by sampling not only if such 
	      event occurs but the failure time of such component is actually sampled.
\end{itemize}

In Case 1, the PWR models in both Dynamic and classical PRA methods are identical: there is a 
one-to-one connection between the stochastic elements (basic events vs. sampled variables) and
accident progression (system simulator vs. ET) of both methods.

\subsection{Stochastic Modeling: Case 1}

In case 1 the set of macro basic events coincide with the set of stochastic parameters sampled
by RAVEN. While in SAPHIRE each macro basic event ($BE_i$) is characterized by probability value 
($p_i$), in RAVEN its corresponding stochastic variable ($v_i$) is modeled through a Bernoulli 
distribution $v_i \sim \operatorname{Bern}(p_i)$.
From this analysis we expect very similar results since the capabilities of Dynamic PRA methods
are not exploited.
However, this is a first test toward benchmarking Dynamic vs. classical PRA methods in order to 
understand modeling differences and impact on final results.


\subsection{Stochastic Modeling:Case 2}

Case 2 expands the Dynamic PRA model by adding time dependent features in the analysis. 
These features extend the stochastic parameters already present in the analysis by quantifying 
timing of events such as failure time of systems.
In addition, recovery time for the LPI system (in both injection and recirculation mode) is 
also included in the analysis.

Thus, timing of events now plays a role in the actual accident progression since timing of 
occurrence of events is directly coupled with system dynamics (i.e., heating-up of the core
due to loss of cooling).

Table~\ref{tab:case2NewVars} shows the additional stochastic variables that have been 
introduced in the analysis and their associated distribution.


\begin{table}
  \caption{Results obtained for Example 2.}
  \label{tab:case2NewVars}
  \centering
  \begin{tabular}{c | c | c | p{5cm}} 
   \hline 
     no. & Name            & Distribution  & Description \\ 
    \hline 
      1  & LPI\_A\_RepTime   &  t   &  Repair time for LPI (Train A) when in injection mode \\
      2  & LPI\_B\_RepTime   &  t   &  Repair time for LPI (Train B) when in injection mode \\
      3  & LPR\_A\_RepTime   &  t   &  Repair time for LPI (Train A) when in recirculation mode \\
      4  & LPR\_B\_RepTime   &  t   &  Repair time for LPI (Train B) when in recirculation mode \\
    \hline 
  \end{tabular}
\end{table}