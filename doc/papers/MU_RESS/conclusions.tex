\section{Conclusions}
\label{sec:conclusions}

In this paper we have presented a first step toward a simulation-based approach to analyze 
multi-unit sites. We have described in detail a method to perform the simulation-based PRA
of a generic multi-unit site by employing codes such
as RAVEN and RELAP5-3D codes. 

The presented analysis exhaustively covered all major steps required to perform a RISMC 
analysis:
\begin{enumerate}
  \item Plant deterministic and stochastic modeling
  \item Plant stochastic analysis
  \item Analysis of results
\end{enumerate}

Regarding Step 1, the considered plant site and accident scenario have been modeled in a great level 
of detail from both a deterministic and stochastic point of view. In particular, the
RAVEN Ensemble Models allowed us to create the links among the six RELAP5-3D models and to
model both system dependencies and timing/sequencing of events at the plant level. 

An important feature of this step is that we have employed ROMs to predict the outcome of 
each unit model (both PWR and SFP). All ROMs have been trained by using large number of simulation 
runs and the ROM prediction has been properly validated. Regarding the 
stochastic modeling, note the presented analysis focused more on the NPP recovery actions while we 
have not introduced additional potential failures of system and components.
   
The plant stochastic analysis has been performed by using classical Monte-Carlo approach. 
This has been a natural choice since the computational time was already decreased by employing 
ROMs instead of the actual codes. A very large number of simulation runs were calculated in 
order to significantly reduce the statistical error of the analysis.

Regarding Step 3, we have employed RAVEN data analysis capabilities in order to analyze the 
large amount of data generated in Step 2. In addition, given the structure of the problem, 
we have created two custom post-processors that could be directly interfaced with RAVEN.

We have presented a detailed analysis of the the generated data; this was accomplished  by employing high 
performance computing systems due to high computational time of each simulation run and due 
to the high number of simulation runs requested. 

We have shown that more quantitative analysis details can be obtained from this kind of 
approach if compared to classical PRA methods that are based on ET/FT algorithms. We were able
to identify how sequencing and timing of events affected the final outcome (i.e., the PDS).
