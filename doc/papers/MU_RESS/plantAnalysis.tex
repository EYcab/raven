\section{Methods of Data Analysis}
\label{sec:plantAnalysisResults}

Historically the concept of CD probability has been typically 
associated to a single unit. At a plant level, a separate value of CD 
probability can be associated to all PWRs and SFPs. However, note that there 
is a high correlation among the six models of the plant site (PWRs and SFPs). 
Thus, a high correlation among CD probabilities of the six models is also expected.

Instead of defining a single CD probability value for each PWR and SFP 
we define a probability value to a Plant Damage State (PDS) variable. This 
variable is a $6$-dimensional vector where each vector element describes the 
status of a plant model. For the scope of this paper we allow two possible 
values to each element of the vector: OK or CD. Hence $2^6=64$ possible 
combinations are allowed.

In order to analyze the data generated by RAVEN we have selected a three steps 
approach:
\begin{enumerate}
  \item Group simulation runs based on their own PDS  
  \item Evaluate probability associated to each PDS and rank PDSs based on 
        their probability values.
  \item Identify communalities that characterize each PDS
\end{enumerate}

In order to perform such analysis, two post-processors were developed in RAVEN in 
addition to the ones already developed within RAVEN itself.
Thus the process of data generation, reduced order modeling and data analysis 
have been fully completed using the RAVEN code.