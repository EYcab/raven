\section{Methods of Data Analysis}
\label{sec:plantAnalysisResults}

Historically the concept of CD probability has been typically 
associated to a single unit. At a plant level, a separate value of CD 
probability can be associated to all PWRs and SFPs. However, note that there 
is a high correlation among the six models of the plant site (PWRs and SFPs). 
Thus, a high correlation among CD probabilities of the six models is also expected.

Instead of defining a single CD probability value for each PWR and SFP 
we define a probability value to a Plant Damage State (PDS) variable. This 
variable is a $6$-dimensional vector where each vector element describes the 
status of a plant model. For the scope of this paper we allow two possible 
values to each element of the vector: OK or CD. Hence $2^6=64$ possible 
combinations are allowed.

In order to analyze the data generated by RAVEN we have selected a three steps 
approach:
\begin{enumerate}
  \item Group simulation runs based on their own PDS  
  \item Evaluate probability associated to each PDS and rank PDSs based on 
        their probability values.
  \item Identify communalities that characterize each PDS
\end{enumerate}

In order to perform such analysis, two post-processors were developed in RAVEN in 
addition to the ones already developed within RAVEN itself.
Thus the process of data generation, reduced order modeling and data analysis 
have been fully completed using the RAVEN code.

\subsection{Error estimation}

Bayesian inference is a method in statistics where not enough data or information is 
available, thus parameter estimates are updated as more information becomes available. 
The results for the simulation are consistently provided as an occurrence of event 
divided by the total number of events to produce a percentage. Instinctually it is 
clear that this type of data corresponds to a beta-binominal distribution.  
Since it is not desired to project the rate of occurrence or probability for each event,
 a non-informative prior is implemented on the beta-binominal distribution [1].
``Some priors are chosen to be non-informative," that is, diffuse enough that they correspond 
to very little prior information. The Jeffreys non-informative prior is often used in this way. 
If information is available, it is more realistic to build that information into the prior, 
but sometimes the information is difficult to find and not worth the trouble. 
In such a case, the Jeffreys non-informative prior can be used [2].''
``The Jeffreys non-informative prior is intended to convey little prior belief or information, 
thus allowing the data to speak for itself. This is useful when no informed consensus exists 
about the true value of the unknown parameter. It is also useful when the prior distribution 
may be challenged by people with various agendas. Some authors use the term reference prior 
instead of "non-informative prior," suggesting that the prior is a standard default, a prior 
that allows consistency and comparability from one study to another.''

The basic equations for this method is as follows:

\begin{equation}
  \alpha_prior = N_i + 0.5  ,  \beta_prior = N_T - N_i + 0.5
  mean = \frac{\alpha_prior}{\beta_prior}
  \label{eq:priorDef}
\end{equation}

where $N_i$ is a specific outcome and $N_T$ is the total number of outcomes for the simulation. 
The parameters of the beta-binominal distribution are then fed to a function that automatically 
produces the $5^{th}$ and $95^{th}$ percentiles of the distribution.  This is reported rather than the 
variance, as variance in a beta-binominal distribution does not adhere to the classic normal 
distribution format of $95^{th}$ confidence interval

