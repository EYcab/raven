Dynamic Probabilistic Risk Analysis (PRA) methods couple stochastic methods 
(e.g., RAVEN, ADAPT, ADS, MCDET) with safety analysis codes (e.g., RELAP5-3D, 
MELCOR, MAAP) to determine risk associated to complex systems such as nuclear 
plants. Compared to classical PRA methods, which are based on static logic 
structures (e.g., Event-Trees, Fault-Trees), they can evaluate with higher 
resolution the safety impact of timing and sequencing of events on the 
accident progression. Recently, special attention has been given to nuclear 
plants which consist of multiple units and, in particular, on the safety 
impact of system dependencies, shared systems and common resources on core 
damage frequencies. In the literature, while classical PRA methods have been 
employed to model multi-unit plants, Dynamic PRA methods have never been applied 
to analyze a full multi-unit model. This paper presents a PRA analysis of a 
multi-unit plant using Dynamic PRA methods. We employ RAVEN as stochastic 
method coupled with RELAP5-3D. The plant under consideration consists of the 
three units and their associated spent fuel pools. The studied initiating
event is a seismic induced station blackout event. We will describe in detail 
how the multi-unit plant has been constructed and, in particular, how unit 
dependencies and shared resources are modeled.