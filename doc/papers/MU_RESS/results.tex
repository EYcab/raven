\section{Results}
\label{sec:results}

We performed a first preliminary analysis of this multi-unit model using a Monte-Carlo sampling. 
We have generated about 2000 simulation runs.  This limited number of simulations cannot be considered 
a sufficient statistical population and, hence, the results obtained can only be considered as preliminary.
A summary of the six more relevant (from a probabilistic point of view) PDSs are shown in Table 2. 
Given the initiating event, there is a probability equal to 21.6 E-3 that the plant at least one model 
will reach a CD situation. In particular, the PWRs of units 2 and 3 are the more sensitive to reach a 
damaged condition. For the PWR of unit 2, the involuntary alignment of the EDGS is the most important 
factor for reaching CD condition.
Note that a PDS that includes more than one model in a CD condition is the PDS number 5 where both the 
PWRs of units 2 and 3 are damaged.
The SFPs can tolerate large time in a SBO condition, however a break in the SFP quickly accelerates 
the heatup process. In this case the first PDS that includes a CD condition in a SFP is the 6th PDS.

\begin{tabular}{lllllllr}
\hline
Rank & \multicolumn{6}{c}{PDS} & Probability           \\
\cline{2-7}
     & PWR1 & PWR2 & PWR3 & SFP1 & SFP2 & SFP3 &       \\
\hline
 1   & OK   & OK   & OK   & OK   & OK   & OK   & 0.6   \\
 2   & OK   & OK   & OK   & OK   & OK   & OK   & 0.4   \\
\hline
\end{tabular}
