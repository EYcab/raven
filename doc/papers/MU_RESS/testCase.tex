\section{Test Case}
\label{sec:testCase}

\subsection{Plant Description}
For the scope of this paper we have chosen a 3-unit plant as shown in Figure 2. 
In more detail, the system we have considered is the following (see Table 1):
\begin{itemize}
  \item Unit 1: 1 PWR (see Figure 3) at full power (100 \%) and it own Spent Fuel Pool (SFP)
  \item Unit 2: 1 PWR in mid-loop operation (i.e., shut-doen mode) and it own SFP. 
                The mid-loop status is characterized by a primary coolant system drained to the 
                hot leg centerline and the existence of openings which a further reduction of 
                the mass inventory poses a serious risk, due to boil off and possible entrainment 
                or spill over of liquid
  \item Unit 3: 1 PWR at full power (108 \%) that restarted a few weeks earlier and its own SFP 
                with a higher heat load since it contains used fuel recently moved from the reactor.
\end{itemize}

In addition, special attention has been given to the design of the electrical and hydraulic systems (see Figure 5):
\begin{itemize}
  \item The plant electrical system is shown in Figure 4. Two electrical switchyards can provide 
        electrical power to all units. All units have a set of Emergency Diesel Generators (EDGs) 
        and, in addition, a swing EDG (i.e., EDGS) can be employed to provide an alternate AC power to either
        Unit 1 or Unit 2. Note also that the 6.6 KV emergengy buses of Unit 1 and Unit 2 can be cross-tied.
  \item The auxiliary feedwtaer (AF) system of Unit 1 and Unit 3 can be cross-tied. Thus cooling to the 
        secondary side can be provided from one unit to the other one.
  \item The Condensate Storage Tanks (CSTs) of Units 2 and Unit 3 can be cross-tied. Thus the water source 
        for the secondary side of either unit can be used as water source for the other one.
\end{itemize}

Plant recovery crew is a shared resource within the plant. As part of the accident secnario, the recovery 
crew can perform AC power and safety injection using mobile equipment located within each unit.


\subsection{Initiaiting Event}

The considered accident scenario is a seisimc event which causes the following events:
\begin{itemize}
  \item Both swityards are disabled
  \item All EDGs are disabled except EDGS which is initially aligned to Unit 2
  \item CST of Unit 2 has lost 80\% of its capacity 
  \item CST of Unit 3 is completely lost
  \item The seisimic event might also rupture the SFPs. Thus a leak might be present during the accident scenario
\end{itemize}

The proposed accident scenario resembles a Station Black Out  (SBO) event at the plant level except for the 
fact that the EDGS is the only source of AC power available and it can be directed toward either Unit 1 or Unit 2.
