\section{Multi-Unit Modeling}
\label{sec:multiUnitModeling}

From a mathematical point of view, a single simulator run can be represented 
as a single trajectory in the phase space. The evolution of such a trajectory 
in the phase space as function of time $t$ can be described as follows:
\begin{equation}
    \frac{\partial \boldsymbol \theta }{\partial t}  = \boldsymbol \Xi (\boldsymbol \theta , \boldsymbol p, \boldsymbol s , t) 
    \label{eq:trajectory}
\end{equation}
where:
\begin{itemize}
  \item $\boldsymbol \theta = \boldsymbol \theta(t)$ represents the temporal 
        evolution of a simulated accident scenario, i.e., $\boldsymbol \theta(t)$ can 
        represent temperature inside the reactor core, the pressure level inside a containment
        building, the radionuclide concentration at a specific point outside the plant, etc.
  \item $\boldsymbol \Xi$ is the actual simulator code that describes how $\boldsymbol \theta$ 
        evolves in time
  \item $\boldsymbol s = \boldsymbol s(t,\boldsymbol p)$ represents the status of components 
        and systems of the model (e.g., status of emergency core cooling system, AC system)
\end{itemize}

By using the RISMC approach, if Monte-Carlo sampling is chosen, the PRA analysis is performed 
by~\cite{BWR_SBO_Mandelli}:
\begin{enumerate}
  \item Associating a probabilistic distribution function (pdf) to the set of uncertain 
        parameters $\boldsymbol p$ (e.g., timing of events)
  \item Performing stochastic sampling of the pdfs defined in Step 1
  \item Performing a simulation run given $\boldsymbol p$ sampled in Step 2, i.e., solve Eq.~\ref{eq:trajectory}
  \item Repeating Steps 2 and 3 $M$ times and evaluating user defined stochastic parameters such as 
        Core Damage (CD) probability $P_{CD}$ as
        \begin{equation}
            P_{CD} = \frac{M_{CD}}{M} 
            \label{eq:CDprobability}
        \end{equation}
        where $M_{CD}$ is the number of simulations that lead to CD. 
\end{enumerate}

In a multi-unit type of scenario, the dynamic behaviour of each unit is not independent but it can actually 
interact with the other units. Example of interactions are: electrical cross-ties and shared plant 
resources such as portable AC generators

Since Equation~\ref{eq:trajectory} refers to a single unit plant site, if multiple units are considered 
then it is needed to track the temporal evolution of each unit,i.e., multiple $\theta$ needs to be evaluated 
(one for each unit). 
Assuming that a three-unit plant is considered, Eq.~\ref{eq:trajectory} now becomes as follows:

\begin{equation}
  \begin{matrix}
     \begin{dcases*}
       \frac{\partial \boldsymbol{\theta}_1}{\partial t}  = \boldsymbol{\Xi}_1 (\boldsymbol{\theta}_1 , \boldsymbol p, \boldsymbol{s}_1 , \boldsymbol{s}_2 , \boldsymbol{s}_3, t)  \\     
       \frac{\partial \boldsymbol{\theta}_2}{\partial t}  = \boldsymbol{\Xi}_2 (\boldsymbol{\theta}_2 , \boldsymbol p, \boldsymbol{s}_1 , \boldsymbol{s}_2 , \boldsymbol{s}_3, t)  \\   
       \frac{\partial \boldsymbol{\theta}_3}{\partial t}  = \boldsymbol{\Xi}_3 (\boldsymbol{\theta}_3 , \boldsymbol p, \boldsymbol{s}_1 , \boldsymbol{s}_2 , \boldsymbol{s}_3, t)  \\       
     \end{dcases*}
  \end{matrix}
  \label{eq:MU_looseCoupled}
\end{equation}

Note that now the vector $s_i (i=1,\ldots,3)$ of each unit is shared among other units. This feature 
captures shared resources and possible system cross-ties among units.
In addition, intra-unit interactions (such as a sub-set of human actions in a unit) may be driven 
by the actual status of other unit (e.g., thermo-hydraulic limit and operational boundaries). 
Again, these actions may have cascade effects on the other units. This is particularly relevant 
for severe accident scenarios. Thus, now Eq.~\ref{eq:MU_looseCoupled} becomes:

\begin{equation}
  \begin{matrix}
     \begin{dcases*}
       \frac{\partial \boldsymbol{\theta}_1}{\partial t}  = \boldsymbol{\Xi}_1 (\boldsymbol{\theta}_1 , \boldsymbol{\theta}_2, \boldsymbol{\theta}_3 , \boldsymbol p, \boldsymbol{s}_1 ,
       \boldsymbol{s}_2 , \boldsymbol{s}_3, t)  \\
       \frac{\partial \boldsymbol{\theta}_2}{\partial t}  = \boldsymbol{\Xi}_2 (\boldsymbol{\theta}_1 , \boldsymbol{\theta}_2, \boldsymbol{\theta}_3 , \boldsymbol p, \boldsymbol{s}_1 ,
       \boldsymbol{s}_2 , \boldsymbol{s}_3, t)  \\
       \frac{\partial \boldsymbol{\theta}_3}{\partial t}  = \boldsymbol{\Xi}_3 (\boldsymbol{\theta}_1 , \boldsymbol{\theta}_2, \boldsymbol{\theta}_3 , \boldsymbol p, \boldsymbol{s}_1 ,
       \boldsymbol{s}_2 , \boldsymbol{s}_3, t)  \\
     \end{dcases*}
  \end{matrix}
  \label{eq:MU_tightCoupled}
\end{equation}

From a modeling point of view, solving Eq.~\ref{eq:MU_looseCoupled} or Eq.~\ref{eq:MU_tightCoupled} poses 
different challenges. Equation~\ref{eq:MU_looseCoupled} can in fact be solved by:
\begin{itemize}
  \item Sampling the set of uncertain parameters $\boldsymbol p$
  \item Determining the temporal profile of $\boldsymbol{s}_1$, $\boldsymbol{s}_2$, $\boldsymbol{s}_3$
  \item Run the simulator for each unit independently given $\boldsymbol p$, $\boldsymbol{s}_1$, $\boldsymbol{s}_2$, $\boldsymbol{s}_3$
\end{itemize}

On the other side, solving Eq.~\ref{eq:MU_tightCoupled} requires a system simulator that allows 
running the simulation of each unit simultaneously and sharing the variables
$\boldsymbol{\theta}_1$, $\boldsymbol{\theta}_2$, $\boldsymbol{\theta}_3$ among them.
This paper will focus on multi-unit case that can be described by Eq.~\ref{eq:MU_looseCoupled}. 



