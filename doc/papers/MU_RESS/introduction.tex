\section{Introduction}
\label{sec:introduction}

Multi-unit Nuclear Power Plant (NPPs) sites are defined as sites that are composed by 
more than one reactor. In the U.S. the situation is the following:
\begin{itemize}
  \item 25 power plant sites have 1 reactor
  \item 33 power plant sites have 2 reactors
  \item 3 power plant sites have 3 reactors
  \item 1 power plant site has 2 reactors and 2 additional reactors under construction
\end{itemize}
The situation is similar for other countries such as Canada and Japan were 
several NPP sites include a large number of reactors (6, 7 or even 8 reactors). 
Worldwide about 80 plants have more than 2 reactors and 32 power plants have more 
than 3 reactors. 

Following the accident event occurred in 2011 at the Fukushima Daiichi~\cite{Fukushima} special 
attention has been given to multi-unit plants. This attention has focused on the 
safety aspects of nuclear reactors that cannot be considered as entities isolated 
from each other. 

Historically, the first multi-unit Probabilistic Risk Analysis (PRA) at the industry
level has been performed for the Seabrook power station~\cite{Seabrook_MU_PRA} using 
classical Probabilistic Risk Assessment (PRA) methods based on Event-Tree (ET) and 
Fault-Trees (FT) tools.
Canada studies have been recently published about CANDU multi-unit 
stations~\cite{CANDU_MU_PRA,Darlington_MU_PRA}.
Furthermore, the analysis of the safety aspects of multi-unit plants has been performed 
for few selected cases ~\cite{MultiUnitKumara,MultiUnitModarres,MultiUnitZhang}. 

The objective of this paper is to propose an analysis of a multi-unit power 
plant not by using classical ET/FT tools~\cite{Nureg1150} but employing a fully 
coupled simulation-based (i.e., Dynamic PRA~\cite{DynamicReliabilityMonteCarlo}) approach: 
the Risk Informed Safety Margin Characterization (RISMC) approach~\cite{RISMC,mandelliNewAlgo}. 

The rationale behind this choice is that great modeling improvements can be achieved by 
employing system simulators instead of static Boolean structures like ETs/FTs. 
Accident dynamics is in fact not set a-priori by the analyst (i.e., in an ET/FT 
structure) but it is entirely simulated given a set of initial and boundary conditions. 
Note that timing and sequencing of events are implicitly modeled in the analysis along 
with interactions between accident evolution and system dynamics.

Modeling limitations of ETs/FT based methods are even more limiting when dealing with 
multiple complex systems that are coupled to each other. In fact, the presence of shared 
systems and structures among the units can add degree of freedom in accident progression 
temporal evolution.
In addition, note that in a multi-unit site, multiple radiological sources are in fact present, 
either nuclear reactors and spent fuel pools (SFPs). 

The paper presents in Section~\ref{sec:rismc} an overview of the RISMC approach and it describes
the PRA elements of the analysis. Section~\ref{sec:raven} presents the RAVEN~\cite{RAVEN_PSAM_2014} 
software and its element that actually enables the construction of the multi-unit modeling. 
Section~\ref{sec:multiUnitModeling} introduces the multi-unit modeling from a mathematical 
point of view while Section~\ref{sec:testCase} describes the multi-unit site considered in the
analysis.
Section~\ref{sec:RISMC_MU_modeling} describes in detail of the modeling of all elements multi-unit
site have been modeled.
An important feature of our analysis is that it employs Reduce Order Models (ROMs) in order to
reduce both the statistical error of the analysis and the computational time. In this respect,
Section~\ref{sec:plantRomModeling} shows how ROMs have been built and validated.
Sections~\ref{sec:multiUnitPRA} and~\ref{sec:plantAnalysisResults} describes how the data have been 
generated and analyzed.
Finally, Section~\ref{sec:results} presents the results obtained by the described RISMC analysis.
 



