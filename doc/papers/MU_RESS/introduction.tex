\section{Introduction}
\label{sec:introduction}

Multi-unit Nuclear Power Plants (NPPs) are defined as plants that are composed by more than one reactor. 
In the U.S. the situation is the following:
\begin{itemize}
  \item 25 power plants have 1 reactor
  \item 33 power plants have 2 reactors
  \item 3 power plants have 3 reactors
  \item 1 power plant has 4 reactors
\end{itemize}
The situation is similar for other countries such as Canada and Japan were 
several power plants include a large number of reactors (6, 7 or even 8 reactors). 
Worldwide about 80 plants have more than 2 reactors and 32 power plants have more than 3 reactors. 

Following the accident event occurred in 2011 at the Fukushima Daiichi special 
attention has been given to multi-unit plants. This attention has focused on the 
safety aspects of nuclear reactors that cannot be considered as entities isolated from each other. 
Historically, the analysis of the safety aspects of multi-unit plants has been performed 
in the past for a few selected cases (Seabrook, Byron/Braidwood) using classical ET/FT tools. 
In addition, more advanced research studies have been developed in~\cite{}. 

The objective of this paper is to propose an analysis of a multi-unit power plant without 
using classical ET/FT tools~\cite{Nureg1150} but employing a fully coupled simulation-based 
(i.e., Dynamic PRA~\cite{DynamicReliabilityMonteCarlo}) approach: 
the RISMC approach~\cite{RISMC,mandelliNewAlgo}. 
The rationale behind this choice is that great modeling improvements can be achieved by 
employing system simulators instead of static Boolean structures like ETs/FTs. 
Accident dynamics is in fact not set a-priori by the analyst (i.e., in an ET/FT structure) 
but it is entirely simulated given a set of initial and boundary conditions. Note that timing and 
sequencing of events are implicitly modeled in the analysis along with interactions between 
accident evolution and system dynamics.
