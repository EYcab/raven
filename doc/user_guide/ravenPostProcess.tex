\subsection{Build RAVEN Input: \xmlNode{PostProcess}}
\label{sub:ravenPostProcess}
The \textbf{PostProcess} step is used to post-process data or manipulate RAVEN entities. It is intended to perform
a single action that is employed by a \textbf{Model} of type \textbf{PostProcessor}. The specification of this
type of step is defined within a \xmlNode{PostProcess} XML block. As for the other objects, the attribute 
\xmlAttr{name} is required and is used to refer to this specific entity in the \xmlNode{RunInfo} block under
the \xmlNode{Sequence} node.

In the \xmlNode{PostProcess} input block, the user needs to specify the objects needed for the different allowable
roles. This step requires the following roles:

\begin{itemize}
  \item \xmlNode{Input}: accepts \textbf{Files}, \textbf{DataObjects} or \textbf{Databases}.
  \item \xmlNode{Model}: Only \xmlString{Models} and \xmlString{PostProcessor} can be assigned to the node's
    attributes \xmlAttr{class} and \xmlAttr{type}, respectively.
  \item \xmlNode{Output}: accepts \textbf{Files}, \textbf{DataObjects}, \textbf{Databases} or \textbf{OutStreams}.
\end{itemize}

As mentioned before, only the model with type \textbf{PostProcessor} is allowed for this step. A post-processor
can be considered as an action performed on a set of data or other type of objects. Most of the post-processors
contained in RAVEN, employ a mathematical operation on the data given as \textbf{``Input''}. Currently, the
following \textbf{PostProcessor} are available in RAVEN:

\begin{itemize}
  \itemsep0em
  \item \textbf{BasicStatistics}
  \item \textbf{ComparisonStatistics}
  \item \textbf{ImportanceRank}
  \item \textbf{SafestPoint}
  \item \textbf{LimitSurface}
  \item \textbf{LimitSurfaceIntegral}
  \item \textbf{External}
  \item \textbf{TopologicalDecomposition}
  \item \textbf{RavenOutput}
  \item \textbf{DataMining}
\end{itemize}

One can use the node attribute \xmlAttr{subType} to select which of the post-processors to be used. As with other
objects, the attribute \xmlAttr{name} is always required so that other RAVEN input XML blocks can use this name
to refer to this specific entity. In addition, each post-processor may require extra sub-nodes, and the user can refer
to the RAVEN user manual for the detailed specifications.

In this example, the \textbf{BasicStatistics} post-processor is used to demonstrate the \textbf{PostProcess} step.
\textbf{BasicStatistics} is a container of the algorithms to compute many of the most important statistical quantities.
Both \textbf{PointSet} and \textbf{HistorySet} can be accepted to compute the static statistics and dynamic statistics,
respectively. In case an \textbf{HistorySet} is provided as \textbf{Input}, the user need to define the \textbf{pivotParameter},
and sometimes the user need to synchronize the \textbf{HistorySet} first via the \textbf{Interfaced} post-processor of
type \textbf{HistorySetSync}.
\xmlExample{framework/user_guide/ravenTutorial/PostProcess.xml}{Models}
In this example, all the metrics of \textbf{BasicStatistics} will be computed for the response $A$.

The \xmlNode{Files} will be used to include all input and output files. In this example, a single input file
for the driven code and two output files of the \textbf{PostProcess} step are defined here. As shown in the following,
two output files are defined for this case study to store the static statistics and dynamic statistics information.
The \xmlString{time} is used as the \xmlNode{pivotParameter}.
\xmlExample{framework/user_guide/ravenTutorial/PostProcess.xml}{Files}


As before, all defined RAVEN entities are combined in the \xmlNode{Steps} block.
\xmlExample{framework/user_guide/ravenTutorial/PostProcess.xml}{Steps}
In this case, three steps have been defined:

\begin{itemize}
  \item \xmlNode{MultiRun} named ``sampleMC'', used to run the multiple instances of the driven code and
     collect the outputs in the two \textit{DataObjects}. As it can be seen, the \xmlNode{Sampler} is inputted
     to communicate to the \textit{Step} that the driven code needs to be perturbed through the Monte Carlo sampling
     strategy.
  \item \xmlNode{PostProcess} named ``statAnalysis\_1'', is used to compute all the statistical moments and FOMs
    based on the data obtained through the sampling strategy. As it can be noticed, the \xmlString{PointSet}
    ``samplesMC'' is used as input to compute the static statistics.
  \item \xmlNode{PostProcess} named ``statAnalysis\_2'', is used to compute all the statistical moments and FOMs
    based on the data obtained through the sampling strategy. As it can be noticed, the \xmlString{HistorySet}
    ``histories'' is used as input to compute the dynamic statistics.
\end{itemize}

%FIXME: The outputs from PostProcess is now stored in the DataObjects, XML format is not supported
%The results from the first \textbf{PostProcess} are listed as follows:
%\xmlExample{framework/user_guide/ravenTutorial/gold/stat/static.xml}{BasicStatisticsPP}
